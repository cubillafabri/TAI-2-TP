%conclusion 
\section{Conclusión}
Para concluir, el código QR es la evolución del código de barras unidimensional, dado que combina escaneo rápido, adaptación al tamaño de los datos, resistencia a errores y daños, alta capacidad de almacenamiento tanto en su dimensión horizontal y vertical. Siendo uno de los métodos para transferir información, no solo para el sector industrial sino también para otras. Esta tecnología se encuentra en la mayoria de los productos de consumo y va apareciendo con más fuerza en otras áreas aumentando la automatización.
\\
Desde su introducción, ha sido un método de entrada, preciso, rápido y con bajo coste en su forma impresa. Los tipos de códigos QR disponibles, permiten gran variedad de aplicaciones en la vida diaria.  Además, las variantes de los códigos bidimensionales muestran las grandes distinciones en su uso y aplicabilidad en determinados sectores, como algunos destacan por su velocidad de lectura sin importar si se encuentran en movimiento, capacidad de codificación, su tamaño mínimo, demostrando que cada uno ofrece algo diferente para un determinado fin.
\\
Sin duda alguna, el símbolo QR se encuentra en muchas áreas algunas son: medicina, marketing y publicidad, entretenimiento, arquitectura y construcción, educación, gastronomia, comercio , entre otros. Cada una de las cuales las utiliza con un propósito diferente, con la capacidad de ofrecer otras formas no ideadas hasta el momento. Sobre todo con el aumento de dispositivos móviles, intentando acelerar su adopción mediante el marketing o el uso en industrias automatizadas.
\\
Sin embargo, si bien existen otras tecnologías capaz de optimizar los tiempos de lectura, con mayor cantidad de modos de funcionamiento, con mayor longitud en cuanto a la transmisión de datos, con mejores presentaciones, y con nuevas mejoras que el código QR no cuenta. A pesar de ello, el código QR permite complementarse con otras tecnologías sin mucha dificultad, siendo ésta en realidad una de sus mejores características.
\\
Finalmente, el código QR necesita una razón que fundamenta porque debería ser escaneada. Un ejemplo de esto sería el uso en la gastronomia, el símbolo es usado para remplazar la pancarta del menú. Actualmente son usados para intercambiar números de telefónos o nombres de usuarios en redes sociales, sin la necesidad de buscarlos en un buscador.