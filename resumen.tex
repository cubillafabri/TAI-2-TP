% Resumen
\begin{abstract}
%The abstract should briefly summarize of the paper in 150--250 words.\\
El código QR, es un código de barra bidimensional, que se ha convertido en uno de los métodos para transferir información, mediante su escaneo por medio de un lector o teléfono móvil. Fue creado en Denso Wave Incorporated siendo una empresa subsidiaria de Toyota, con el próposito de gestionar los inventarios de piezas, en las plantas de fabricación de automóviles. Sus características principales son la alta velocidad de decodificación, el bajo coste del decodificador, facilidad de lectura, capacidad de almacenamiento, resistencia a errores y daños, entre otros.  Los tipos de códigos QR, incluyen las mejoras del modelo 1 al modelo 2, la reducción de un símbolo a un cuadrado de 1mm -Micro Código-, las ventajas del código iQR, la encriptación de datos por medio del SQRC y la personalización disponible por el código frame QR. Además, las variantes existentes y sus respectivas ventajas y desventajas con respecto al código QR. Por último, examinaremos la variedad de usos y aplicaciones del símbolo QR en el mundo real y, cuales son las tecnologías que compiten o se complemetan con la tecnología QR.  

\begin{keywords} 
   Código QR \and Respuesta Rápida \and Código bidimensional \and Código unidimensional \and SQRC \and iQR \and Micro Código QR \and Reed-Solomon \and PDF417 \and MaxiCode \and Datamatrix \and RFID \and NFC \and BLE beacons \and AR \and Image Recognition \and LinkRay
\end{keywords}

\end{abstract}