% definicion del codigo qr
\section{Código QR}
El código QR (Quick Response o Respuesta Rápida), es un código de barra bidimensional (2D) que contiene información tanto en la dimensión horizontal como en la vertical. El código QR combina escaneo rápido, alta capacidad de almacenamiento, tamaño pequeño y a menudo se denominan códigos de barras, independientemente de si están compuesto por barras, cuadrados u otros elementos con forma. \cite{2012_DENSO}
Su expresión 2D de información permite escribir información en gran volumen y alta densidad. Además, son un tipo de matriz bidimensional parecido a un tablero de ajedrez en donde la información se expresa colocando celdas blancas y negras. El código QR para difundirlo, se declaró dominio público (un código sin reclamos de patente).  \cite{2019_Hara}
\begin{figure}
\centering
\qrcode{} 
\caption{Ejemplo de un Código QR}
\label{fig:qrcodeexample}
\end{figure}

Desde su introducción en 1994, el código QR se ha convertido en uno de los métodos para transferir información de algo impreso a lo digital. En particular son muy útiles para los usuarios de teléfonos móviles; la información de anuncios, productos, empaques, ficha técnica de peliculas , URLs y muchos otros, se transmitiran al teléfono móvil. El código QR se desarrolló anteriormente solo para el sector industrial, pero se ha convertido en un foco de estrategia en la publicidad y el empaque del consumidor en los últimos años, con un acceso rápido hacia la marca en cuestión. También pueden estar vinculados con una ubicación para determinar el lugar del escaneo del código, o la aplicación que escanea recupera la geoinformación usando GPS o la URL codificado en él.\cite{2015_Emran_BOOK,2012_Varallyai}