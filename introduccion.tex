%introduccion
\section{Introducción}
Desde sus inicios, el código de barras se ha convertido en una de las formas de acelerar el proceso de pago, control del stock de mercancías, etiquetado de precios a productos, automatización del registro y el seguimiento de los productos en las pequeñas o grandes empresas con un gasto mínimo. Además, están en todas partes, identificando o ubicando todo lo que se mueva de forma novedosa e inteligente. 
\\
La tecnología a progresado de forma ingeniosa obteniendo mejoras durante varias décadas, desde que se patento el primer código de barras; con dificultades para su adopción, luego de varios años de intentos, el comite sobre el código uniforme de productos comestibles recomendo usar en la mayoría de los productos.  Luego de que aparecieron los teléfonos móviles, teniendo la capacidad de codificación y decodificación han adquirido de vuelta una gran popularidad, tanto que han aparecido códigos de barras personalizables, con encriptación de los datos y con diferentes tamaños para utilizarlos en donde sea necesario.
\\
El código QR es capaz de contener más información que la tecnología anterior, y últimamente tiene un uso muy extendido en casi todas las áreas existentes, incluido el uso con otras tecnologías más avanzadas. Se estima que se escanean 6.000 millones de veces al día, y que están presentes en productos, anuncios, pasaportes, entre otros. 